% Options for packages loaded elsewhere
\PassOptionsToPackage{unicode}{hyperref}
\PassOptionsToPackage{hyphens}{url}
\PassOptionsToPackage{dvipsnames,svgnames,x11names}{xcolor}
\documentclass[
  12pt,
]{article}
\usepackage{xcolor}
\usepackage[margin=1in]{geometry}
\usepackage{amsmath,amssymb}
\setcounter{secnumdepth}{-\maxdimen} % remove section numbering
\usepackage{iftex}
\ifPDFTeX
  \usepackage[T1]{fontenc}
  \usepackage[utf8]{inputenc}
  \usepackage{textcomp} % provide euro and other symbols
\else % if luatex or xetex
  \usepackage{unicode-math} % this also loads fontspec
  \defaultfontfeatures{Scale=MatchLowercase}
  \defaultfontfeatures[\rmfamily]{Ligatures=TeX,Scale=1}
\fi
\usepackage{lmodern}
\ifPDFTeX\else
  % xetex/luatex font selection
\fi
% Use upquote if available, for straight quotes in verbatim environments
\IfFileExists{upquote.sty}{\usepackage{upquote}}{}
\IfFileExists{microtype.sty}{% use microtype if available
  \usepackage[]{microtype}
  \UseMicrotypeSet[protrusion]{basicmath} % disable protrusion for tt fonts
}{}
\makeatletter
\@ifundefined{KOMAClassName}{% if non-KOMA class
  \IfFileExists{parskip.sty}{%
    \usepackage{parskip}
  }{% else
    \setlength{\parindent}{0pt}
    \setlength{\parskip}{6pt plus 2pt minus 1pt}}
}{% if KOMA class
  \KOMAoptions{parskip=half}}
\makeatother
\usepackage{longtable,booktabs,array}
\usepackage{calc} % for calculating minipage widths
% Correct order of tables after \paragraph or \subparagraph
\usepackage{etoolbox}
\makeatletter
\patchcmd\longtable{\par}{\if@noskipsec\mbox{}\fi\par}{}{}
\makeatother
% Allow footnotes in longtable head/foot
\IfFileExists{footnotehyper.sty}{\usepackage{footnotehyper}}{\usepackage{footnote}}
\makesavenoteenv{longtable}
\usepackage{graphicx}
\makeatletter
\newsavebox\pandoc@box
\newcommand*\pandocbounded[1]{% scales image to fit in text height/width
  \sbox\pandoc@box{#1}%
  \Gscale@div\@tempa{\textheight}{\dimexpr\ht\pandoc@box+\dp\pandoc@box\relax}%
  \Gscale@div\@tempb{\linewidth}{\wd\pandoc@box}%
  \ifdim\@tempb\p@<\@tempa\p@\let\@tempa\@tempb\fi% select the smaller of both
  \ifdim\@tempa\p@<\p@\scalebox{\@tempa}{\usebox\pandoc@box}%
  \else\usebox{\pandoc@box}%
  \fi%
}
% Set default figure placement to htbp
\def\fps@figure{htbp}
\makeatother
\setlength{\emergencystretch}{3em} % prevent overfull lines
\providecommand{\tightlist}{%
  \setlength{\itemsep}{0pt}\setlength{\parskip}{0pt}}
\usepackage{setspace}
\onehalfspacing
\usepackage{bookmark}
\IfFileExists{xurl.sty}{\usepackage{xurl}}{} % add URL line breaks if available
\urlstyle{same}
\hypersetup{
  pdftitle={Virginia's Longevity Divide: Income, Obesity, and Exercise Access Across Counties},
  pdfauthor={Team REGAN},
  colorlinks=true,
  linkcolor={Maroon},
  filecolor={Maroon},
  citecolor={Blue},
  urlcolor={black},
  pdfcreator={LaTeX via pandoc}}

\title{Virginia's Longevity Divide: Income, Obesity, and Exercise Access
Across Counties}
\author{Team REGAN}
\date{}

\begin{document}
\maketitle

\begin{center}

\includegraphics[width=\textwidth]%
  {D:/OneDrive - University of Virginia/Will M/Undergraduate Research Courses-Grants/Summer 2025/Introduction to Regression Analysis (STAT 3220)/Final Project/3220-Final-Project/Graphics/Title_Page/TitlePageImage.png}
  
\end{center}

\newpage

\subsection{Introduction}\label{introduction}

Although life expectancy is often regarded as a concise indicator of a
community's overall well-being, notable disparities still exist even
within a single state. Virginia's 133 counties and independent cities
have an average life expectancy of just over seventy-seven years,
according to the 2025 County Health Rankings report, highlighting
disparities in the state's progress toward public health objectives
(County Health Rankings \& Roadmaps, 2025). Three recurrent factors that
influence longevity are household economic resources, the burden of
chronic diseases (obesity), and the characteristics of local
environments (such as rural versus urban areas), according to decades of
epidemiological research, including the Centers for Disease Control and
Prevention's findings on social determinants of health (Hacker, 2022)
and the World Health Organization's analyses of obesity-related
mortality (WHO, 2025). Understanding how these factors manifest at the
county level in Virginia can guide targeted interventions and equitable
allocation of resources.

The overarching question guiding this study is:

\emph{Which socioeconomic, health-behavior, and environmental factors
best explain the variation in 2025 life expectancy across Virginia's
counties and independent cities?}

Drawing on publicly available data from the County Health Rankings
portal, we assembled a dataset in which each row represents a single
jurisdiction, and the outcome of interest is the average life expectancy
in years.

Three specific research questions are presented to focus the general
investigation. First, \emph{is the average life expectancy longer in
jurisdictions with greater median household incomes?} This tackles the
quantifiable relationship that previous national studies have proposed
between longevity and economic prosperity. Second, \emph{is life
expectancy lower in counties with the highest tertile of adult obesity
rates than in those with the lowest tertile?} This investigates the
relationship between mortality risk and a quantifiably modifiable health
behavior variable. Third, \emph{is there a difference in mean life
expectancy between Virginia's primarily rural and urban jurisdictions?}
This question investigates whether geographic context alone confers a
longevity advantage or disadvantage by contrasting a qualitative
classification of place.

\subsection{Data Summary}\label{data-summary}

\subsubsection{Data Sources}\label{data-sources}

This analysis is based on the 2025 Virginia extract from the
\emph{County Health Rankings \& Roadmaps program}, a collaborative
effort between the University of Wisconsin Population Health Institute
and the Robert Wood Johnson Foundation that gathers county-level health
indicators nationwide each year. Before harmonizing the series to a
single 2025 reference year, the institute gathers each metric directly
from approved federal sources, including the Bureau of Labor Statistics,
the American Community Survey, and the CDC WONDER database.

For this study, the population is the complete set of 133 Virginia
counties and independent cities. Two companion tables provided by the
Rankings, Select Measure Data and Additional Measure Data, contain (i)
core health outcomes and (ii) socioeconomic and health-behavior
covariates. These tables were combined on each county's five-digit FIPS
code to create a single cross-sectional data frame. A continuous
response variable---life expectancy at birth (years)---was retained
exactly as reported, ensuring comparability with CDC methodology.

Several small, logically motivated changes were necessary. The ratio of
primary care physicians was provided as a text string, such as
``2,210:1.'' To ensure that lower figures indicate increased provider
availability, it was transformed to a straightforward numerical count of
residents per physician and then split into 5 bins ``\textless=1k'',
``1-2k'', ``2-3k'', ``3-5k'', ``\textgreater=5k''. To draw attention to
non-linear gradients in built-environment resources, the percentage of
people with access to exercise opportunities was recoded into an ordered
three-level factor (Low, Mid, and High). Ultimately, the rural
population percentage was divided into four equal-width groups, referred
to as rural bands: 0--25\%, 26--50\%, 51--75\%, and 76--100\%. This
discretization facilitates the understanding of regional differences
while maintaining a monotonic ordering. After removing occasional
entries with missing outcome values and one jurisdiction without a
county name, 132 observations were included in the study sample.

The County Health Rankings have a high level of credibility, as they are
frequently referenced in peer-reviewed health services research and
utilize open documentation of data provenance and imputation procedures.
However, three cautions are worth mentioning. First, particular data
show suppressed numbers for a limited number of counties, which can
contribute to the uncertainty. Second, there is no set time for
retrieval because several focus areas of data are gathered in different
years. Lastly, measurements are obtained from a wide range of reliable
sources, each with its own unique data collection techniques, and then
compiled into sheets. These restrictions will be reviewed when
evaluating model findings; however, they do not compromise the dataset's
overall integrity.

\subsubsection{Exploratory Data
Analysis}\label{exploratory-data-analysis}

\begin{verbatim}
Warning: Removed 1 row containing non-finite outside the scale range
(`stat_bin()`).
\end{verbatim}

\pandocbounded{\includegraphics[keepaspectratio]{UPDATED_STAT3220_Report_Template_files/figure-latex/unnamed-chunk-2-1.pdf}}

\begin{verbatim}
Warning: Removed 1 row containing non-finite outside the scale range
(`stat_smooth()`).
\end{verbatim}

\begin{verbatim}
Warning: Removed 1 row containing missing values or values outside the scale range
(`geom_point()`).
\end{verbatim}

\pandocbounded{\includegraphics[keepaspectratio]{UPDATED_STAT3220_Report_Template_files/figure-latex/unnamed-chunk-2-2.pdf}}
\pandocbounded{\includegraphics[keepaspectratio]{UPDATED_STAT3220_Report_Template_files/figure-latex/unnamed-chunk-2-3.pdf}}

\begin{verbatim}
Warning: Removed 1 row containing non-finite outside the scale range (`stat_smooth()`).
Removed 1 row containing missing values or values outside the scale range
(`geom_point()`).
\end{verbatim}

\pandocbounded{\includegraphics[keepaspectratio]{UPDATED_STAT3220_Report_Template_files/figure-latex/unnamed-chunk-2-4.pdf}}
\pandocbounded{\includegraphics[keepaspectratio]{UPDATED_STAT3220_Report_Template_files/figure-latex/unnamed-chunk-2-5.pdf}}
\pandocbounded{\includegraphics[keepaspectratio]{UPDATED_STAT3220_Report_Template_files/figure-latex/unnamed-chunk-2-6.pdf}}
\pandocbounded{\includegraphics[keepaspectratio]{UPDATED_STAT3220_Report_Template_files/figure-latex/unnamed-chunk-2-7.pdf}}

\subsubsection{EDA Summary}\label{eda-summary}

\pandocbounded{\includegraphics[keepaspectratio]{UPDATED_STAT3220_Report_Template_files/figure-latex/unnamed-chunk-3-1.pdf}}
\pandocbounded{\includegraphics[keepaspectratio]{UPDATED_STAT3220_Report_Template_files/figure-latex/unnamed-chunk-3-2.pdf}}
\pandocbounded{\includegraphics[keepaspectratio]{UPDATED_STAT3220_Report_Template_files/figure-latex/unnamed-chunk-3-3.pdf}}
The response variable, life expectancy at birth, has a mean of 75.15
years, a standard deviation of only 3.89 years, and a bell-shaped
distribution that is almost symmetric with a slight skew (--0.04). The
variable fulfills the normality and homoscedasticity assumptions
typically required for multiple linear regression, as it spans more than
twenty-four years (64.3--88.9); however, the majority of counties
cluster firmly between 73 and 78 years. As a result, no transformation
seems to be required.

A substantial, positive correlation, which steadily plateaus above USD
90,000, is evident when life expectancy is plotted against median
household income. Together, the LOESS fit (residual SE = 2.33 years) and
the Pearson correlation of 0.75 suggest that economic improvement in
lower-income areas may result in significant longevity benefits.
Extremely wealthy regions, on the other hand, already seem to be close
to a ceiling. On the other hand, the adult obesity rate exhibits a
linear, detrimental effect: a 0.68-year drop in life expectancy is
predicted for every percentage point rise, and the correlation of --0.72
accounts for more than half of the variation (R2 = 0.52). When obesity
is re-examined within tertiles of access to exercise opportunities, its
relationship with longevity steepens in high-access environments (r =
--0.79, slope = --0.84), suggesting that behavioral choices, rather than
structural constraints, exacerbate health disparities in environments
with plentiful facilities.

Provider availability also matters, though more subtly. Counties with
fewer than 1,000 residents per primary-care physician record a median
expectancy of 75.9 years, while mid-access bands (2,000--5,000 residents
per doctor) drop to 74.8--75.2 years before a slight rebound in the
scarcest group. This pattern suggests that physician density exerts
incremental benefits only up to a threshold, beyond which contextual
factors---such as wealth, built environment, or rurality---likely
dominate.

The geographic context itself is significant. The most extended
lifespans (median = 78.2 years) are found in jurisdictions with 26--50\%
rural populations, outliving both highly urban counties (74.8 years) and
the most rural groups (= 75.5 years). While larger dispersion within
metropolitan areas (SD = 5.65) indicates a mix of impoverished
inner-city populations and wealthy suburbs, semi-rural regions tend to
be more socioeconomically homogeneous and generally healthier. The
correlation heat-map clarifies potential collinearity. Age-adjusted
death rate is nearly a mirror image of the outcome itself (r = --0.98);
therefore, it will be excluded from formal modelling to avoid
redundancy. Income and obesity correlate at --0.63, a level that may
inflate variance inflation factors but not so high as to demand outright
removal; instead, diagnostics will verify tolerable VIF values. Other
predictors---physician ratio, rurality, and exercise access---display
only modest intercorrelations (\textbar r\textbar{} \textless{} 0.29),
indicating that they each convey largely distinct information about
county environments.

When considered collectively, the exploratory study demonstrates that
multiple linear regression with life expectancy as the continuous
response is appropriate. Additionally, it supports a principled
variable-screening approach that limits the age-adjusted death rate for
conceptual and statistical reasons, monitors multicollinearity primarily
between income and obesity, and keeps income, obesity, exercise access,
rural band, and physician ratio as core predictors. The finished model
will be well-positioned to explain why some Virginians live noticeably
longer than others, as it will capture both quantitative gradients and
categorical contexts.

\subsection{Methods and Analysis}\label{methods-and-analysis}

\subsection{Results}\label{results}

\subsection{Conclusions}\label{conclusions}

\newpage

\subsection{Appendix A: Data
Dictionary}\label{appendix-a-data-dictionary}

\begin{longtable}[]{@{}ccc@{}}
\toprule\noalign{}
Variable Name & Abbreviated Name & Description \\
\midrule\noalign{}
\endhead
\bottomrule\noalign{}
\endlastfoot
& & \\
\end{longtable}

\newpage

\subsection{Appendix B: Data Rows}\label{appendix-b-data-rows}

\newpage

\subsection{Appendix C: Final Model Output and
Plots}\label{appendix-c-final-model-output-and-plots}

\newpage

\subsection{Appendix D: References}\label{appendix-d-references}

\subsubsection{Background}\label{background}

\subsubsection{Data Sources}\label{data-sources-1}

\subsubsection{Additional Help}\label{additional-help}

\end{document}
